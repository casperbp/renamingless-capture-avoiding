\documentclass[a4paper,UKenglish,cleveref, autoref, thm-restate]{oasics-v2021}

%for A4 paper format use option "a4paper", for US-letter use option "letterpaper"
%for british hyphenation rules use option "UKenglish", for american hyphenation rules use option "USenglish"
%for section-numbered lemmas etc., use "numberwithinsect"
%for enabling cleveref support, use "cleveref"
%for enabling autoref support, use "autoref"
%for anonymousing the authors (e.g. for double-blind review), add "anonymous"
%for enabling thm-restate support, use "thm-restate"
%for enabling a two-column layout for the author/affilation part (only applicable for > 6 authors), use "authorcolumns"
%for producing a PDF according the PDF/A standard, add "pdfa"

\usepackage{mathtools}
\usepackage{stmaryrd}
\usepackage{wasysym}
\usepackage{agda}
\usepackage{agdadimmed}
\usepackage{newunicodechar}

%\pdfoutput=1 %uncomment to ensure pdflatex processing (mandatatory e.g. to submit to arXiv)
%\hideOASIcs %uncomment to remove references to OASIcs series (logo, DOI, ...), e.g. when preparing a pre-final version to be uploaded to arXiv or another public repository

\bibliographystyle{plainurl}% the mandatory bibstyle

\makeatletter
\newcommand{\crefnames}[3]{%
  \@for\next:=#1\do{%
    \expandafter\crefname\expandafter{\next}{#2}{#3}%
  }%
}
\makeatother

\crefnames{section}{\S}{\S\S}

%%
%% Agda typesetting commands shorthands, for
%% manual typesetting of inline code
%%

\newcommand{\af}{\AgdaFunction}
\newcommand{\un}{\AgdaUnderscore}
\newcommand{\ad}{\AgdaDatatype}
\newcommand{\ab}{\AgdaBound}
\newcommand{\ac}{\AgdaInductiveConstructor}
\newcommand{\aF}{\AgdaField}
\newcommand{\as}{\AgdaSymbol}
\newcommand{\ak}{\AgdaKeyword}
\newcommand{\ap}{\AgdaPrimitiveType}
\newcommand{\an}{\AgdaNumber}
\newcommand{\aC}{\AgdaComment}
\newcommand{\am}{\AgdaModule}

\setlength{\parskip}{0em} 
\setlength\mathindent{0.2cm}

% \newcommand{\citep}[1]{\cite{#1}}
% \newcommand{\citet}[1]{\citeauthor{#1}~\cite{#1}}


%%
%% Unicode for typesetting Agda code
\input{unicode.tex}

%% Title information
\title{Renamingless Capture-Avoiding Substitution, Intrinsically Scoped}

%% Author with single affiliation.
\author{Casper {Bach Poulsen}}{Delft University of Technology, Netherlands \and \url{http://www.casperbp.net} }{c.b.poulsen@tudelft.nl}{https://orcid.org/0000-0003-0622-7639}{}%TODO mandatory, please use full name; only 1 author per \author macro; first two parameters are mandatory, other parameters can be empty. Please provide at least the name of the affiliation and the country. The full address is optional. Use additional curly braces to indicate the correct name splitting when the last name consists of multiple name parts.
%
\authorrunning{C. Bach Poulsen}
% First names are abbreviated in the running head.
% If there are more than two authors, 'et al.' is used.

\Copyright{Jane Open Access and Joan R. Public} %TODO mandatory, please use full first names. LIPIcs license is "CC-BY";  http://creativecommons.org/licenses/by/3.0/

\begin{CCSXML}
<ccs2012>
   <concept>
       <concept_id>10011007.10011006.10011039.10011311</concept_id>
       <concept_desc>Software and its engineering~Semantics</concept_desc>
       <concept_significance>500</concept_significance>
       </concept>
   <concept>
       <concept_id>10003752.10003790.10002990</concept_id>
       <concept_desc>Theory of computation~Logic and verification</concept_desc>
       <concept_significance>300</concept_significance>
       </concept>
 </ccs2012>
\end{CCSXML}

\ccsdesc[500]{Software and its engineering~Semantics}
\ccsdesc[300]{Theory of computation~Logic and verification}

\keywords{Capture-avoiding substitution, Untyped lambda calculus, Agda, Dependent types} %TODO mandatory; please add comma-separated list of keywords

%\category{} %optional, e.g. invited paper

%\relatedversion{} %optional, e.g. full version hosted on arXiv, HAL, or other respository/website
%\relatedversiondetails[linktext={opt. text shown instead of the URL}, cite=DBLP:books/mk/GrayR93]{Classification (e.g. Full Version, Extended Version, Previous Version}{URL to related version} %linktext and cite are optional

%\supplement{}%optional, e.g. related research data, source code, ... hosted on a repository like zenodo, figshare, GitHub, ...
%\supplementdetails[linktext={opt. text shown instead of the URL}, cite=DBLP:books/mk/GrayR93, subcategory={Description, Subcategory}, swhid={Software Heritage Identifier}]{General Classification (e.g. Software, Dataset, Model, ...)}{URL to related version} %linktext, cite, and subcategory are optional

%\funding{(Optional) general funding statement \dots}%optional, to capture a funding statement, which applies to all authors. Please enter author specific funding statements as fifth argument of the \author macro.

%\acknowledgements{I want to thank \dots}%optional

%\nolinenumbers %uncomment to disable line numbering

%Editor-only macros:: begin (do not touch as author)%%%%%%%%%%%%%%%%%%%%%%%%%%%%%%%%%%
\EventEditors{John Q. Open and Joan R. Access}
\EventNoEds{2}
\EventLongTitle{42nd Conference on Very Important Topics (CVIT 2016)}
\EventShortTitle{CVIT 2016}
\EventAcronym{CVIT}
\EventYear{2016}
\EventDate{December 24--27, 2016}
\EventLocation{Little Whinging, United Kingdom}
\EventLogo{}
\SeriesVolume{42}
\ArticleNo{23}
%%%%%%%%%%%%%%%%%%%%%%%%%%%%%%%%%%%%%%%%%%%%%%%%%%%%%%

\begin{document}

\maketitle

\begin{abstract}
  We describe a simple and direct technique for capture avoiding substitution of untyped $\lambda$ terms that avoids the need to rename bound variables during substitution.
  We demonstrate how this substitution technique yields correct normalization of open $\lambda$ terms to weak head normal form.
  We also give an intrinsically scoped syntax for untyped $\lambda$ terms.
  Using this syntax we show that the substitution technique is, indeed, capture avoiding.
\end{abstract}

\section{Introduction}

\cite{}

%%% Local Variables:
%%% reftex-default-bibliography: ("../references.bib")
%%% End:

\input{sections/2-interpretation.tex}
\input{sections/3-normalization.tex}
\section{Related Work}
\label{sec:related}

In this paper we explored two techniques for capture avoiding substitution that avoids renaming, for the purpose of implementing static name binding in languages with $\lambda$s.
The topic of evaluating $\lambda$ expressions has a long and rich history.
Summarizing it all is beyond the scope of this paper; for overviews see, e.g., the works of Barendregt~\cite{DBLP:journals/bsl/Barendregt97} or Cardone~and~Hindley~\cite{hindley-lambda-history}.
We discuss a few of the papers that are most closely related to the techniques we have described.

In their formalization of $\lambda$ calculus and type theory, McKinna~and~Pollack~\cite{DBLP:journals/jar/McKinnaP99} consider a system that uses named substitution without renaming, for a particular notion of open terms.
They consider a syntax that distinguishes two classes of names: \emph{parameters} and \emph{variables}.
\emph{Variable substitution} does not affect parameters, and \emph{parameter substitution} does not affect variables.
Their notion of variable substitution is defined for terms that are \emph{variable-closed}, but which may be \emph{parameter-open}.
Thus, by encoding free variables as parameters, their system can be used to compute with open terms.
However, syntactically distinguishing free variables this way seems to presupposes a static binding analysis.
The approach we discussed in \cref{sec:interpreting-open} does not presuppose such static analysis.

Our paper considers how to interpret open terms.
There exist several calculi in the literature for evaluating open terms.
Accatolli~and~Guerrieri~\cite{DBLP:conf/aplas/AccattoliG16} gives an overview of several of these calculi for \emph{open call-by-value}, which is the class of languages that the interpreters in \cref{sec:interpretation}~and\cref{sec:shifting} interpret.
In their paper, Accatolli~and~Guerrieri focus on the meta-theory of these calculi.
For their meta-theoretical study they rely on an unspecified notion of capture-avoiding substitution.
In this paper, we explore how to define such capture-avoiding substitution in a way that does not perform renaming.
While the meta-theory of Berkling-Fehr substitution has been studied~\cite{berkling1982amodification}, the meta-theory of the substitution strategy in~\cref{sec:interpreting-open} remains an open question.

\section{Conclusion}
\label{sec:conclusion}

We have discussed two techniques for implementing capture avoiding substitution in definitional interpreters in a way that does not require renaming of bound variables.
One of the techniques relies on a coarse-grained but simple discipline for closing terms, is known to not support interpretation strategies that evaluate under binders, and has (to the best of our knowledge) not been studied meta-theoretically.
The other technique is an existing technique from the literature.
While this technique is less efficient, it is more fine-grained and so does is not subject to the same limitations as the first technique.
It also has a well-established meta-theory. 
Neither of the two techniques seem to be widely known or at least not widely applied.
With this work, we hope to increase awareness of these techniques.

%%% Local Variables:
%%% reftex-default-bibliography: ("../references.bib")
%%% End:



%% Bibliography
\bibliography{references}



%% Appendix
%% \appendix
%% \section{Appendix}
%% 
%% Text of appendix \ldots
%% 
\end{document}
