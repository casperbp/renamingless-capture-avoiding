\section{Related Work}
\label{sec:related}

In this paper we explored two techniques for capture avoiding substitution that avoids renaming, for the purpose of implementing static name binding in languages with $\lambda$s.
The topic of evaluating $\lambda$ expressions has a long and rich history.
Summarizing it all is beyond the scope of this paper; for overviews see, e.g., the works of Barendregt~\cite{DBLP:journals/bsl/Barendregt97} or Cardone~and~Hindley~\cite{hindley-lambda-history}.
We discuss a few of the papers that are most closely related to the techniques we have described.

In their formalization of $\lambda$ calculus and type theory, McKinna~and~Pollack~\cite{DBLP:journals/jar/McKinnaP99} consider a system that uses named substitution without renaming, for a particular notion of open terms.
They consider a syntax that distinguishes two classes of names: \emph{parameters} and \emph{variables}.
\emph{Variable substitution} does not affect parameters, and \emph{parameter substitution} does not affect variables.
Their notion of variable substitution is defined for terms that are \emph{variable-closed}, but which may be \emph{parameter-open}.
Thus, by encoding free variables as parameters, their system can be used to compute with open terms.
However, syntactically distinguishing free variables this way seems to presupposes a static binding analysis.
The approach we discussed in \cref{sec:interpreting-open} does not presuppose such static analysis.

Our paper considers how to interpret open terms.
There exist several calculi in the literature for evaluating open terms.
Accatolli~and~Guerrieri~\cite{DBLP:conf/aplas/AccattoliG16} gives an overview of several of these calculi for \emph{open call-by-value}, which is the class of languages that the interpreters in \cref{sec:interpretation}~and~\cref{sec:shifting} interpret.
Accatolli~and~Guerrieri focus on the meta-theory of these calculi.
To this end, they rely on an unspecified notion of capture-avoiding substitution.
In this paper, we explore how to implement such capture-avoiding substitution functions in interpreters in a way that does not perform renaming.
% While the meta-theory of Berkling-Fehr substitution has been studied~\cite{berkling1982amodification}, the meta-theory of the substitution strategy in~\cref{sec:interpreting-open} remains an open question.


\section{Conclusion}
\label{sec:conclusion}

We have discussed two techniques for implementing capture avoiding substitution in definitional interpreters in a way that does not require renaming of bound variables.
One of the techniques relies on a coarse-grained but simple discipline for closing terms which works well for interpretation strategies that do not evaluate under binders.
The other technique, due to Berkling~and~Fehr~\cite{berkling1982amodification}, is more fine-grained and also works for interpretation strategies that evaluate under binders.
While less expressive, the former technique is simpler to implement, and is slightly more efficient.
Neither of the two techniques seem to be widely known or at least not widely applied.
With this work, we hope to increase awareness of these techniques.

%%% Local Variables:
%%% reftex-default-bibliography: ("../references.bib")
%%% End:
